\documentclass{beamer}
%\documentclass[trans]{beamer}

\usetheme[tub,nosubsections]{qu}

\title{An Open-Source C\texttt{++} Framework for Multithreaded Realtime
Multichannel Audio Applications}

\author{%
Matthias Geier\inst{1} \and Torben Hohn\inst{2} \and Sascha Spors\inst{1}}

\institute{\inst{1}%
Quality and Usability Lab\\
Telekom Innovation Laboratories\\
Technische Universit\"at Berlin
\and\inst{2}%
Linutronix GmbH}

\date[LAC 2012]{Linux Audio Conference\\
April 15\textsuperscript{th}, 2012}

\subject{Linux Audio Conference (LAC) 2012}
\keywords{MIMO, realtime, multithreading, C\texttt{++}}

\graphicspath{{images/}}

\usepackage{listings}

\newcommand{\code}{\texttt}

\newcommand{\distance}{1.2em}
\usepackage{tikz}
\usetikzlibrary{positioning,calc}

\tikzset{>=stealth} % nice arrow heads
\tikzstyle{box}=[draw,rectangle,font=\sffamily]
\tikzstyle{nobox}=[font=\em\sffamily,inner sep=.3ex]
\tikzstyle{round}=[circle,draw,inner sep=0pt]
\tikzstyle{point}=[circle,draw,fill,inner sep=0pt,minimum size=4\pgflinewidth]
\tikzstyle{arrow}=[draw,->,shorten >=\pgflinewidth]
\tikzstyle{noarrow}=[draw,-,shorten >=0pt]

\tikzstyle{every edge}+=[arrow]

\tikzstyle{fork}=[{
grow'=east,
anchor=west,
child anchor=west,
%parent anchor=east,
%sibling distance=\distance,
level distance=\distance,
edge from parent path=[arrow](\tikzparentnode\tikzparentanchor) |- (\tikzchildnode\tikzchildanchor)}]

\tikzset{node distance=\distance and \distance}

\begin{document}

% 25 - 30 minutes

\maketitle

\begin{frame}{APF}{Audio Processing Framework}

\begin{itemize}
\item collection of C\texttt{++} code
\item \url{http://tu-berlin.de/?id=apf}
\item open source, GPLv3
\item mostly platform-independent
\begin{itemize}
\item tested on Linux and MacOSX
\end{itemize}
\item quite generic
\item heavy use of templates
\item mostly header-only
\end{itemize}

\end{frame}

\begin{frame}{APF Components}
\begin{itemize}
\item \code{MimoProcessor} {\color{TUBblue}$\to$ \emph{topic of this talk!}}
\end{itemize}

\begin{itemize}
\item biquad \& cascade of biquads (IIR filter)
\begin{itemize}
\item several methods for denormal prevention
\end{itemize}
\item (yet another) C\texttt{++} wrapper for JACK
\item iterators
\item other tools
\end{itemize}

\begin{itemize}
\item delay line {\color{TUBblue}$\to$ \emph{coming soon!}}
\item partitioned convolution (FIR filter) {\color{TUBblue}$\to$ \emph{coming soon!}}
\end{itemize}
\end{frame}

\section{\code{MimoProcessor}}

\begin{frame}{\code{MimoProcessor}}{Target Applications}
\begin{itemize}
\item block-based audio applications
\item many inputs/outputs
\item both realtime and non-realtime
\item interactive applications
\end{itemize}
\vfill
\pause
example application areas:
\begin{itemize}
\item sound field synthesis
\item multichannel echo cancelling
\item beamforming
\end{itemize}
\end{frame}

\def\fork#1{
\path[fork] node[point] (fork#1) {}
[sibling distance=4.5ex,every child node/.style=box]
child foreach \number in {1,...,4} {node (h-#1-\number) {$h_{#1,\number}(t)$}}
node [left=of fork#1,nobox] (input#1) {input #1} edge[noarrow] (fork#1);
}

\section{Example}

\begin{frame}{\code{MimoProcessor} Example Application}{Generic MIMO system}

\def\shift{-19ex}

\begin{center}
\begin{tikzpicture}
\footnotesize

\fork{1}

\begin{scope}[yshift=\shift]
\fork{2}
\end{scope}

\foreach \n in {1,...,4}
{
\node [node distance=3*\distance,right=of h-1-\n,yshift=.5*\shift,round] (plus-\n) {+};
\node [right=of plus-\n,nobox] (output-\n) {output \n};
\path (h-1-\n.east) edge (plus-\n);
\path (h-2-\n.east) edge (plus-\n);
\path (plus-\n) edge (output-\n);
}

\end{tikzpicture}
\end{center}
\end{frame}

\section{Features}

\begin{frame}{\code{MimoProcessor} Features}
\begin{itemize}
\item different audio backends (realtime and non-realtime)
\begin{itemize}
\item \emph{JACK}
\item \emph{Pure Data} {\color{TUBblue}(and \emph{Max/MSP})} external using \emph{flext}
\item \emph{MEX}-file for \emph{octave} {\color{TUBblue}(and \emph{Matlab})}
\item read/write multichannel audio files
\item \emph{PortAudio} {\color{TUBblue}$\to$ \emph{coming soon!}}
\item \dots and users can provide their own backends!
\end{itemize}
\pause
\item ``automatic'' multithreading
\item optional crossfade
\item dynamic number of inputs/outputs
  {\color{TUBblue}(if supported by the audio backend)}
\item safe communication between realtime and non-realtime threads
\end{itemize}
\end{frame}

\begin{frame}{Realtime \& Non-Realtime Threads}
\begin{itemize}
\item realtime thread:
\begin{itemize}
\item audio callback function
\item limited computation time per audio block
\item blocking functions are \alert{forbidden}, e.g.
\begin{itemize}
\item allocating/deallocating memory
\item reading/writing files/sockets
\item creating/joining threads
\item waiting for mutexes
\item \dots
\end{itemize}
\end{itemize}
\pause
\vfill
\item non-realtime thread:
\begin{itemize}
\item GUI, network, reading/writing files
\item everything else
\end{itemize}
\end{itemize}
\end{frame}

\begin{frame}{MimoProcessor Components}
\begin{itemize}[<+->]
\item \code{LockFreeFifo<Command*>}
\begin{itemize}[<.->]
\item \code{push()}, \code{pop()}
\end{itemize}
\item \code{CommandQueue}
\begin{itemize}[<.->]
\item using $2\times$ \code{LockFreeFifo}
\item \code{push()}/\code{wait()}, \code{process\_commands()}
\end{itemize}
\item \code{Command}
\begin{itemize}[<.->]
\item abstract base class
\item \code{execute()}, \code{cleanup()}
\end{itemize}
%\item \code{SharedData<T>}
%\begin{itemize}[<.->]
%\item getter/setter, using \code{CommandQueue}
%\end{itemize}
\item \code{RtList<Item*>}
\begin{itemize}[<.->]
\item using \code{CommandQueue}
\item \code{add()}/\code{rem()}, \code{begin()}/\code{end()}/\code{size()}
\end{itemize}
\item \code{Item}
\begin{itemize}[<.->]
\item abstract base class
\item \code{process()}
\end{itemize}
\end{itemize}
\end{frame}

\begin{frame}{MimoProcessor Components}
\begin{itemize}
\item \code{MimoProcessor<Derived,} \emph{policies see below} \code{>}
\pause
\begin{itemize}
\item \code{interface\_policy}
\begin{itemize}
\item \code{jack\_policy}
\item \code{pointer\_policy}
\end{itemize}
\pause
\item \code{thread\_policy}
\begin{itemize}
\item \code{posix\_thread\_policy}
\end{itemize}
\pause
\item \code{sync\_policy}
\begin{itemize}
\item \code{posix\_sync\_policy}
\end{itemize}
\pause
\item \code{xfade\_policy}
\begin{itemize}
\item \code{raised\_cosine\_policy} (default)
\item \code{disable\_xfade}
\end{itemize}
\end{itemize}
\end{itemize}
\end{frame}

\section{Code, yay!}

\lstset{language=[ISO]C++}
\lstset{basicstyle=\scriptsize\ttfamily} 
\lstset{commentstyle=\itshape\color{TUBblue}}
\lstset{keywordstyle=\bfseries\color{TUBred}}
\lstset{basewidth=0.54em}
%\lstset{numbers=left,numberstyle=\tiny\color{TUBdarkgray},numbersep=1ex}

\begin{frame}[fragile=singleslide]{Code Example}
{\code{examples/jack\_minimal.cpp}}
% includes and MyProcessor class definition
\lstinputlisting[firstline=26,lastline=47]{jack_minimal.cpp}
\end{frame}

\begin{frame}[fragile=singleslide,t]
% MyProcessor::Output implementation
\lstinputlisting[firstline=49,lastline=72]{jack_minimal.cpp}
\end{frame}

\begin{frame}[fragile=singleslide,t]
% MyProcessor constructor and main()
\lstinputlisting[firstline=74,lastline=95]{jack_minimal.cpp}
\end{frame}

\section{Parallel Processing}

\begin{frame}{Parallel Processing}
\begin{itemize}[<+->]
\item \code{RtList<Item*>}: list of polymorphic base class pointers
\item virtual function \code{Item::process()}
\item items within one list are processed in parallel
\item fixed number of threads, specified by user
\item simple scheduling:
\begin{itemize}
\item<.-> each of the $N$ threads gets every $N$-th item
\end{itemize}
\item one ``main audio thread'', $N-1$ ``worker threads''
\begin{itemize}
\item<.-> communication via semaphores
\end{itemize}
\end{itemize}
\end{frame}

\section{X-Fade}

\begin{frame}{Crossfade}
\begin{itemize}
\item block-based processing
\item parameter changes only at block boundaries
\item artifacts due to discontinuities
\item can be reduced by crossfade
\end{itemize}
\pause

\begin{itemize}
\item each block is processed twice
\begin{enumerate}
\item with previous parameters, fade out
\item with current parameters, fade in
\end{enumerate}
\item but: only if something actually changes
\begin{itemize}
\item as noticed by \code{CommandQueue}
\end{itemize}
\item crossfade is optional
\begin{itemize}
\item can be switched off at compile time
\begin{itemize}
\item \code{MimoProcessor<\dots, disable\_xfade>}
\end{itemize}
\end{itemize}
\end{itemize}
\end{frame}

\section{Example}

\begin{frame}{Example}{Near-Field-Compensated Higher Order Ambisonics}
\begin{itemize}
\item implementation of a realtime NFC-HOA renderer
\item circular loudspeaker array (2.5D)
\item $M$-th order ($2\times M + 1$ loudspeakers)
\pause
\item stages of the algorithm (in \texttt{RtList}s):
\begin{description}[$N\times (M+1)$~]
\pause
\item[$N$~] sources/inputs
\pause
\item[$N\times (M+1)$~] IIR filters
\pause
\item[$M+1$~] objects which add contributions of sources per order
\begin{itemize}
\item multiplication with 2 complex weighting factors
\item resulting in $2\times M + 1$ values (per audio sample)
\end{itemize}
\pause
\item[block size~] IFFTs {\color{TUBblue}(of length $2\times M + 1$)}
\pause
\item[$2\times M + 1$~] outputs
\end{description}
\pause
\item part of the SSR {\color{TUBblue}$\to$ \emph{coming soon!}}
\begin{itemize}
\item What is the SSR? {\color{TUBblue}$\to$ see next page}
\end{itemize}
\end{itemize}
\end{frame}

\section{SSR}

\begin{frame}{The SoundScape Renderer (SSR)}
\begin{itemize}
\item software tool for \emph{object-based} realtime spatial audio reproduction
\item several different reproduction methods
\begin{itemize}
\item Binaural Renderer
\item Binaural Room Synthesis (BRS)
\item Wave Field Synthesis (WFS)
\item Vector Base Amplitude Panning (VBAP)
\item Ambisonic Amplitude Panning (AAP)
\item Generic Renderer
\item Binaural Playback Renderer (BPB)
\item NFC-HOA Renderer {\color{TUBblue}$\to$ \emph{coming soon!}}
\end{itemize}
\end{itemize}

\pause

\begin{itemize}
\item runs on Linux and Mac OSX
\item uses the \emph{Jack Audio Connection Kit} (JACK)
\item graphical user interface (Qt) and network interface (TCP/IP)
\item Free and Open Source Software (GPLv3)
\item \url{http://tu-berlin.de/?ssr}
\end{itemize}
\end{frame}

{
\setbeamertemplate{background canvas}
[vertical shading][top=white,middle=TUBdarkgray,bottom=white]
\setbeamertemplate{background}{%
\begin{tikzpicture}[remember picture, overlay]
\node[anchor=north east,yshift=-0.5cm] at (current page.north east)
{\includegraphics[scale=.27]{mozart_full_muted_wfs}};
\node[anchor=south west] at (current page.south west)
{\includegraphics[scale=.27]{mozart_full_muted_binaural}};
\end{tikzpicture}
}
\leftlogo{}\author{}\title{}
\begin{frame}
\frametitle{\hspace{-.3333em}\tikz[baseline=(x.base)] \node[fill=white,fill opacity=.6,text opacity=1,text depth=0pt] (x) {The SoundScape Renderer (SSR).};}
\framesubtitle{Graphical User Interface.}
% no frame text
\end{frame}
}

\section{Conclusion}

\begin{frame}{Important Notes}

\begin{itemize}
\item compile with optimization!
\begin{itemize}
\item e.g. \code{g++ -O3}
\end{itemize}
\end{itemize}
\pause

\begin{itemize}
\item be aware of cache effects!
\begin{itemize}
\item memory locality
\item false sharing
\end{itemize}
\item look for bottlenecks with a profiler!
\begin{itemize}
\item OProfile, gprof, \dots
\end{itemize}
\end{itemize}

\end{frame}

\begin{frame}{Conclusion}

\begin{itemize}
\item goal of \code{MimoProcessor}: to be \dots
\begin{itemize}
\item unobtrusive
\item easy to use
\item re-usable in different contexts
\item easily extensible (e.g. by policy-based design)
\end{itemize}
\end{itemize}
\pause

\begin{itemize}
\item parallelization: simple yet effective
\begin{itemize}
\item trade-off between effort {\color{TUBblue}(to use, to implement)} and performance
\item significant gain in performance, e.g. for HOA renderer
\end{itemize}
\end{itemize}
\pause

\begin{itemize}
\item unit tests are included
\begin{itemize}
\item using the \emph{CATCH} framework
\end{itemize}
\item well documented
\begin{itemize}
\item \emph{Doxygen} documentation also available at the website
\end{itemize}
\end{itemize}

\end{frame}

\begin{frame}{Future Work}
\begin{itemize}
\item Audio Processing Framework (APF)
\begin{itemize}
\item include delay line and partitioned convolution
\item implement \emph{PortAudio} policy
\end{itemize}
\end{itemize}

\begin{itemize}
\item SoundScape Renderer (SSR)
\begin{itemize}
\item re-write core using the \code{MimoProcessor}
\item port all existing renderers
\item include the brand-new NFC-HOA renderer
\end{itemize}
\end{itemize}
\end{frame}

\section{~}

{
% don't count last frame and don't show frame number
\setbeamercolor{page number in head/foot}{fg=page number in head/foot.bg}
\addtocounter{framenumber}{-1}
\begin{frame}[b]
\setbeamercolor{thanks}{bg=TUBgray}
\centerline{%
\begin{beamercolorbox}[wd=.6\paperwidth,sep=3mm,center,rounded=true]{thanks}
Thank you very much for your attention!\\
\alert{Questions?}
\end{beamercolorbox}}
\vspace{2\baselineskip}
\begin{center}
Website: \url{http://tu-berlin.de/?apf}\\
%e-Mail: \texttt{\insertemail}\\
Blog: \url{http://audio.qu.tu-berlin.de}
\end{center}
\vspace{\baselineskip}
\end{frame}
}

%\appendix

\end{document}
