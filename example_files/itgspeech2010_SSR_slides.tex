\documentclass{beamer}
%\documentclass[trans]{beamer}
\usetheme{qu}

\title{Conducting Psychoacoustic Experiments\\
with the \emph{SoundScape Renderer}}

\author{Matthias Geier \and Sascha Spors}
\emaillink{SoundScapeRenderer@telekom.de}

\date[ITG speech 2010]{ITG Fachtagung Speech Communication\\
October 8\textsuperscript{th}, 2010}

\usepackage{tikz}
\usetikzlibrary{chains,positioning,calc}

\graphicspath{{images/}}

\begin{document}

\maketitle

\section{SSR}

\subsection{Features}

\begin{frame}{The SoundScape Renderer (SSR).}
%\tikz[remember picture,overlay] \node[anchor=north east,scale=.8]
%at (current page.north east) {\includegraphics{ssr_logo}};
\begin{itemize}
\item software tool for \emph{object-based} real-time spatial audio reproduction
\item several different \hyperlink{reproductionmethods}{reproduction methods}
\begin{itemize}
\item for headphone-based and loudspeaker-based systems
\end{itemize}
\item can be used as audio back-end for experiments
\end{itemize}

\pause

\begin{itemize}
\item runs on Linux
\begin{itemize}
\item port to Mac OS X planned
\end{itemize}
\item uses the \emph{Jack Audio Connection Kit} (JACK)
\item Free and Open Source Software (GPL license)
\item \url{http://tu-berlin.de/?id=ssr}
%\item \texttt{\insertemail}
\end{itemize}
\end{frame}

\subsection{GUI}

{
\setbeamertemplate{background canvas}
[vertical shading][top=TeleWeiss,middle=TeleGrau1,bottom=TeleWeiss]
%\setbeamercolor{lower separation line head}{bg=TeleSchwarz}

\setbeamertemplate{background}{%
\begin{tikzpicture}[remember picture, overlay]
\node[anchor=north east,yshift=-0.3cm] at (current page.north east)
{\includegraphics[scale=.26]{screenshot_wfs}};
\node[anchor=south west] at (current page.south west)
{\includegraphics[scale=.26]{screenshot_binaural}};
\end{tikzpicture}
}
\leftlogo{}\author{}\title{}
\begin{frame}
\frametitle{\hspace{-.3333em}\tikz[baseline=(x.base)] \node[fill=white,fill opacity=.6,text opacity=1,text depth=0pt] (x) {The SoundScape Renderer (SSR).};}
\framesubtitle{Graphical User Interface.}
% no frame text
\end{frame}
}

\subsection{Network Interface}

\begin{frame}[fragile]{The SoundScape Renderer (SSR).}{Network Interface.}
\begin{itemize}
\item several TCP/IP connections at once are possible
\item you can use your favorite tools to connect:
\begin{itemize}
\item Python, Matlab, Pure Data, Flash/ActionScript, Ruby, Tcl/Tk, Java, \dots
\item example implementations available
\end{itemize}
\end{itemize}

\pause

\begin{itemize}
\item example network message:
\footnotesize
\begin{semiverbatim}
<request>
  <source id="4" volume="-6">
    <position x="1.5" y="1"/>
  </source>
  <source id="7" mute="true"/>
</request>\grau\\0
\end{semiverbatim}
\end{itemize}
\end{frame}

\section{Rendering Modules}

\begin{frame}{The SoundScape Renderer (SSR).}{Modular Architecture.}
\label<2>{reproductionmethods}
\begin{columns}[t]
\column{.3\textwidth}	
\textbf{DSP Blocks:}
\begin{itemize}
\item Delay
\item Filter (partitioned convolution)
\item Weight
\end{itemize}
\column<2->{.55\textwidth}	
\textbf{Rendering Modules:}
\begin{itemize}
\item \hyperlink{binaural}{Binaural Renderer}
\item \hyperlink{wfs}{Wave Field Synthesis (WFS)}
\item \hyperlink{vbap}{Vector Base Amplitude Panning (VBAP)}
\item \hyperlink{aap}{Ambisonic Amplitude Panning (AAP)}
\item \hyperlink{brs}{Binaural Room Synthesis (BRS)}
\item \hyperlink{generic}{Generic Renderer}
\item \hyperlink{bpb}{Binaural Playback Renderer (BPB)}
\end{itemize}
\end{columns}
\end{frame}

% TikZ settings for all pictures:
\newcommand{\distance}{1.2em}
\newcommand{\paralleldistance}{.7ex}
\tikzset{>=stealth} % nice arrow heads
\tikzstyle{box}=[draw,rectangle,font=\sffamily,text depth=0pt]
%\tikzstyle{every node}+=[fill=yellow]
%\tikzstyle{every node}+=[semithick]
\tikzstyle{nobox}=[font=\em\sffamily,inner sep=.3ex]
\tikzstyle{round}=[circle,draw,inner sep=0pt]
\tikzstyle{point}=[circle,draw,fill,inner sep=0pt,minimum size=4\pgflinewidth]
\tikzstyle{arrow}=[draw,->,shorten >=\pgflinewidth]
\tikzstyle{every join}+=[arrow]
\tikzstyle{every edge}+=[arrow]
\tikzstyle{chainjoin}=[on chain, join]
\tikzstyle{noarrowchainjoin}=[on chain,join={by -,shorten >=0pt}]
\tikzstyle{fork}=[{grow=east,sibling distance=\distance,level distance=\distance,edge from parent path=[arrow](\tikzparentnode\tikzparentanchor) |- (\tikzchildnode\tikzchildanchor)}]
\tikzset{node distance=\distance and \distance}

% arguments to \genfrac: opening delimiter, closing del., width of line, style
% style: 0 \displaystyle, 1 \textstyle, 2 \scriptstyle, 3 \scriptscriptstyle
% long story short: to change the size, change the optional argument!
\newcommand{\twolines}[3][1]{$\genfrac{}{}{0pt}{#1}
{\text{\vphantom{Xg}#2}}{\text{\vphantom{Xg}#3}}$}

\newcommand{\clipnode}[1]{%
% 'even odd rule' must be active!
% TODO maybe something like that is already defined?
\clip
(#1.south west) rectangle (#1.north east)
(current bounding box.south west) rectangle (current bounding box.north east);}

\subsection{Binaural}

\begin{frame}{Binaural Renderer.}
\label<3>{binaural}

\begin{center}
\begin{tikzpicture}[start chain,even odd rule]
\node[nobox,chainjoin] {\twolines{source}{signal}};
\node[box,chainjoin]   (weight)  {Weight};
\path[fork] node[point,noarrowchainjoin] {}
child {node [box,anchor=north west] (filterR) {Filter}}
child {node [box,anchor=south west] (filterL) {Filter}};

\chainin (filterL);
\node[round,chainjoin] (plusL) {$+$};
\coordinate[chainjoin] (endL);

\chainin (filterR);
\node[round,chainjoin] (plusR) {$+$};
\coordinate[chainjoin] (endR);

% place the headphones node midway between endR and endL
\path (endR) -- node[nobox,xshift=.6em] {\twolines{head-}{phones}} (endL);

\pause

\node[nobox,above=of weight]  {\twolines{source--listener}{distance}}
  edge (weight);
\node[nobox,above=of filterL] (angle of incidence)
  {\twolines{angle of}{incidence}};

\begin{scope} % for clipping
\clipnode{filterL}
\path let
\p1=(angle of incidence.south),
\p2=(\x1-.5*\paralleldistance,\y1),\p3=(\x1+.5*\paralleldistance,\y1) in
(\p2) edge (\p2 |- filterL.north)
(\p3) edge (\p3 |- filterR.north);
\end{scope}

\pause

\node[nobox,above=of plusL] {\twolines{other}{sources}} edge (plusL);
\node[nobox,below=of plusR] {\twolines{other}{sources}} edge (plusR);
\end{tikzpicture}
\end{center}

\onslide
\begin{itemize}
\item one set of \emph{head-related impulse responses} (HRIRs)
\item typically non-reverberant
\item with head tracking
\end{itemize}

\end{frame}

\subsection{BRS}

\begin{frame}{Binaural Room Synthesis Renderer (BRS).}
\label<3>{brs}

\begin{center}
\begin{tikzpicture}[start chain,even odd rule]
\node[nobox,chainjoin] {\twolines{source}{signal}};
\path[fork] node[point,noarrowchainjoin] {}
child {node [box,anchor=north west] (filterR) {Filter}}
child {node [box,anchor=south west] (filterL) {Filter}};

\chainin (filterL);
\node[round,chainjoin] (plusL) {$+$};
\coordinate[chainjoin] (endL);

\chainin (filterR);
\node[round,chainjoin] (plusR) {$+$};
\coordinate[chainjoin] (endR);

% place the headphones node midway between endR and endL
\path (endR) -- node[nobox,xshift=.6em] {\twolines{head-}{phones}} (endL);

\pause

\node[nobox,above=of filterL] (head rotation)
  {\twolines{head}{rotation}};

\begin{scope} % for clipping
\clipnode{filterL}
\path let
\p1=(head rotation.south),
\p2=(\x1-.5*\paralleldistance,\y1),\p3=(\x1+.5*\paralleldistance,\y1) in
(\p2) edge (\p2 |- filterL.north)
(\p3) edge (\p3 |- filterR.north);
\end{scope}

\pause

\node[nobox,above=of plusL] {\twolines{other}{sources}} edge (plusL);
\node[nobox,below=of plusR] {\twolines{other}{sources}} edge (plusR);
\end{tikzpicture}
\end{center}

\onslide
\begin{itemize}
\item each source uses its own set of \emph{binaural room impulse responses}
	(BRIRs)
\begin{itemize}
\item measurements of real rooms and real loudspeakers
\item simulation of virtual loudspeaker setups
\end{itemize}
\item source position is implicitly contained in BRIRs
\item with head tracking
\end{itemize}

\end{frame}

\subsection{WFS}

\begin{frame}{Wave Field Synthesis Renderer (WFS).}
\label<3>{wfs}

\begin{center}
\begin{tikzpicture}[start chain,remember picture]
\node[nobox,chainjoin] {\twolines{source}{signal}};
\node[box,chainjoin]   (filter)  {Filter};
\node[point,noarrowchainjoin] (point) {};

\node[box,chainjoin]   (delay)   {Delay};
\node[box,chainjoin]   (weight)  {Weight};
\node[round,chainjoin] (plus) {$+$};
\node[nobox,chainjoin] {\twolines{loud-}{speaker}};

\pause

\node[nobox,above=of delay,xshift=-1ex] {\twolines{source--loudspeaker}{distance}}
  edge (delay) edge (weight);
\node[nobox,above=of weight,xshift=1ex] {\twolines{angle of}{incidence}}
  edge (weight);

\pause

\chainin (point);
\node[nobox,on chain=going below,join] {\twolines{other}{loudspeakers}};

\node[nobox,below=of plus] {\twolines{other}{sources}} edge (plus);
\end{tikzpicture}
\end{center}

\onslide
\begin{itemize}
\item driving function is split up into three parts
\item pre-filter to compensate source type mismatch
	($2\frac{1}{2}$-dimensional reproduction)
\item only one filter per source, efficient implementation
\item arbitrary concave loudspeaker setups can be used
\item sources can be placed in front of loudspeakers (= focused sources)
\end{itemize}

\end{frame}

\subsection{Generic}

\begin{frame}{Generic Renderer.}
\label<2>{generic}

\begin{center}
\begin{tikzpicture}[start chain]
\node[nobox,chainjoin] {\twolines{source}{signal}};
\node[point,noarrowchainjoin] (point) {};
\node[box,chainjoin]   (filter) {Filter};
\node[round,chainjoin] (plus) {$+$};
\node[nobox,chainjoin] {\twolines{loud-}{speaker}};

\pause

\chainin (point);
\node[nobox,on chain=going below,join] {\twolines{other}{loudspeakers}};

\node[nobox,below=of plus] {\twolines{other}{sources}} edge (plus);
\end{tikzpicture}
\end{center}

\onslide
\begin{itemize}
\item one filter for each combination of source and loudspeaker
\item implicitly contained in the filters:
\begin{itemize}
\item source position
\item loudspeaker position
\item reproduction method
\end{itemize}
\end{itemize}

\end{frame}

\subsection{BPB}

\begin{frame}{Binaural Playback Renderer (BPB).}
\label<3>{bpb}

\begin{center}
\begin{tikzpicture}[start chain,even odd rule]
\node[box,chainjoin] (playerL) {Player};
\node[round,chainjoin] (plusL) {$+$};
\coordinate[chainjoin] (endL);

\node [box,below=of playerL] (playerR) {Player};
\chainin (playerR);
\node[round,chainjoin] (plusR) {$+$};
\coordinate[chainjoin] (endR);

% place the headphones node midway between endR and endL
\path (endR) -- node[nobox,xshift=.6em] {\twolines{head-}{phones}} (endL);

\pause

\node[nobox,above=of playerL] (head rotation)
  {\twolines{head}{rotation}};

\begin{scope} % for clipping
\clipnode{playerL}
\path let
\p1=(head rotation.south),
\p2=(\x1-.5*\paralleldistance,\y1),\p3=(\x1+.5*\paralleldistance,\y1) in
(\p2) edge (\p2 |- playerL.north)
(\p3) edge (\p3 |- playerR.north);
\end{scope}

\pause

\node[nobox,above=of plusL] {\twolines{other}{sources}} edge (plusL);
\node[nobox,below=of plusR] {\twolines{other}{sources}} edge (plusR);
\end{tikzpicture}
\end{center}

\onslide
\begin{itemize}
\item for a one degree resolution, 720-channel sound files are necessary
%\item no block processing artifacts
\item especially useful for:
\begin{itemize}
\item rapidly moving sources
\item algorithms without real-time implementation
\end{itemize}
\end{itemize}

\end{frame}

\section{Experiment Design}

\begin{frame}{Experiment Design.}{Make it modular!}
\begin{itemize}
\item modules
\begin{itemize}
\item user interface (computer screen, keyboard, MIDI-controller, \dots)
\item file player (optional)
\item spatial renderer (SSR)
\item network interface to connect everything
\end{itemize}
\end{itemize}

\begin{itemize}
\item offline tasks
\begin{itemize}
\item generate lists of stimuli and conditions
\item analyze and visualize the results
\end{itemize}
\end{itemize}
\end{frame}

\subsection{GUI}

{
\setbeamertemplate{background canvas}
[vertical shading][top=TeleWeiss,middle=TeleGrau1,bottom=TeleWeiss]
%\setbeamercolor{lower separation line head}{bg=TeleSchwarz}
\author{}\title{}

\begin{frame}{Test GUI (example).}
\vspace{\logomargin}
\noleftmargin\hspace{\logomargin}\includegraphics[scale=.46]{screenshot_rating}
\begin{tikzpicture}[remember picture,overlay]
\node[yshift=-3mm,anchor=north east,inner sep=\logomargin] at (current page.north east)
{\includegraphics[scale=.46]{screenshot14}};
\end{tikzpicture}
\end{frame}
}

\subsection{Setup}

\begin{frame}{Test Setup (example).}
\tikzstyle{big box}=[box,minimum height=3em,minimum width=4em]
\tikzstyle{arrow label}=[font=\tiny,inner sep=.1em]
\begin{center}
\begin{tikzpicture}
\uncover<2->{\node[big box] (player) {Player};}
\node[big box,right=4em of player] (ssr) {SSR};
\node[nobox,right=of ssr] (headphones) {\twolines{head-}{phones}};

\path \foreach \m in {-.5*\paralleldistance,.5*\paralleldistance}
{
  ([yshift=\m]ssr.east) edge ([yshift=\m]headphones.west)
};

\node[nobox,above=of ssr] {\twolines{head}{tracker}} edge (ssr);

\node[big box,anchor=north,yshift=-2em] at ($(player.south)!.5!(ssr.south)$)
(gui) {GUI};
\alt<1>{
\path (gui) edge[out=70,in=-110] node[arrow label,auto=right]
{TCP/IP} (ssr);}{
\path (gui) edge[dashed,out=70,in=-110] node[arrow label,auto=right]
{TCP/IP} (ssr);}

\uncover<2->{
\path (gui) edge[out=110,in=-70] node[arrow label,auto=left] {TCP/IP}
(player);
\path let \n1=7 in
\foreach \k in {1,...,\n1}
{
let \n2={(\n1/2+0.5-\k)*\paralleldistance} in
([yshift=\n2]player.east) edge ([yshift=\n2]ssr.west)
};
\path (player.north east) -- node[arrow label]{audio channels} (ssr.north west);
}

\node[nobox,left=of gui] {\twolines{list of}{stimuli}} edge (gui);
\node[nobox,right=of gui] (logfile) {\twolines{log}{file}};
\path (gui) edge (logfile);
\end{tikzpicture}
\end{center}
\end{frame}

\section{~} % empty (but clickable) title for summary

\begin{frame}{Summary.}
\begin{columns}[t]
\column{.4\textwidth}
%\begin{itemize}
%\item modular test design
\textbf{Modular Test Design}
\begin{itemize}
\item user interface
\item spatial renderer
\item audio file player
\item connected by network
\end{itemize}
%\end{itemize}

\column{.5\textwidth}
%\begin{itemize}
%\item SoundScape Renderer
\textbf{SoundScape Renderer}
\begin{itemize}
\item object-based paradigm
\item many rendering methods/algorithms
\begin{itemize}
\item for headphones and loudspeakers
\end{itemize}
\item remote control by network interface
\item freely available (GPL)
\end{itemize}
%\end{itemize}
\end{columns}
\end{frame}

{
\setbeamertemplate{headline}{}
\begin{frame}[b]
\setbeamercolor{thanks}{bg=TeleGrau6}
\centerline{%
\begin{beamercolorbox}[wd=.5\paperwidth,sep=3mm,center,rounded=true]{thanks}
  Thank you very much for your attention!\\
  \alert{Questions?}
\end{beamercolorbox}}

\vspace{2\baselineskip}
%\vfill

\begin{center}
Download: \url{http://tu-berlin.de/?id=ssr}\\
e-Mail: \texttt{\insertemail}\\
Blog: \url{http://audio.qu.tu-berlin.de}
\end{center}

\vspace{\baselineskip}
\end{frame}
}

\appendix

\section{Appendix:}

\frame{\centerline{Appendix}}

\section{More Renderers}

\subsection{VBAP}

\begin{frame}{Vector Base Amplitude Panning Renderer (VBAP).}
\label<3>{vbap}

\begin{center}
\begin{tikzpicture}[start chain,even odd rule]
\node[nobox,chainjoin] {\twolines{source}{signal}};
\node[box,chainjoin]   (weight)  {Weight};
\path[fork] node[point,noarrowchainjoin] {}
child {node [box,anchor=north west] (weightR) {Weight}}
child {node [box,anchor=south west] (weightL) {Weight}};

\chainin (weightL);
\node[round,chainjoin] (plusL) {$+$};
\coordinate[chainjoin] (endL);

\chainin (weightR);
\node[round,chainjoin] (plusR) {$+$};
\coordinate[chainjoin] (endR);

% place the output node midway between endR and endL
\path (endR) -- node[nobox,xshift=1em] {\twolines{pair of}{loudspeakers}} (endL);

\pause

\node[nobox,above=of weight] {\twolines{source--loudspeaker}{distance}}
  edge (weight);
\node[nobox,above=of weightL] (angle of incidence)
  {\twolines{angle of}{incidence}};

\begin{scope} % for clipping
\clipnode{weightL}
\path let
\p1=(angle of incidence.south),
\p2=(\x1-.5*\paralleldistance,\y1),\p3=(\x1+.5*\paralleldistance,\y1) in
(\p2) edge (\p2 |- weightL.north)
(\p3) edge (\p3 |- weightR.north);
\end{scope}

\pause

\node[nobox,above=of plusL] {\twolines{other}{sources}} edge (plusL);
\node[nobox,below=of plusR] {\twolines{other}{sources}} edge (plusR);
\end{tikzpicture}
\end{center}

\onslide
\begin{itemize}
\item panning function between closest loudspeaker pair
\begin{itemize}
\item only two loudspeakers are active for each source
\end{itemize}
\item no time delay
\end{itemize}

\end{frame}

\subsection{AAP}

\begin{frame}{Ambisonic Amplitude Panning Renderer (AAP).}
\label<3>{aap}

\begin{center}
\begin{tikzpicture}[start chain]
\node[nobox,chainjoin] {\twolines{source}{signal}};
\node[point,noarrowchainjoin] (point) {};
\node[box,chainjoin]   (weight)  {Weight};
\node[round,chainjoin] (plus) {$+$};
\node[nobox,chainjoin] {\twolines{loud-}{speaker}};

\pause

\node[nobox,above=of weight,anchor=south east] {\twolines{source--loudspeaker}{distance}}
  edge (weight);
\node[nobox,above=of weight,anchor=south west] {\twolines{angle of}{incidence}}
  edge (weight);

\pause

\chainin (point);
\node[nobox,on chain=going below,join] {\twolines{other}{loudspeakers}};

\node[nobox,below=of plus] {\twolines{other}{sources}} edge (plus);
\end{tikzpicture}
\end{center}

\onslide
\begin{itemize}
\item panning function for potentially all loudspeakers
\end{itemize}

\end{frame}

\end{document}
