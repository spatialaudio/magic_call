\documentclass[a4paper]{article}

\usepackage{itgspeech2010}    %% Include ITGSpeech2010 style
\usepackage{times}            %% Choose Times Roman font
%\usepackage[ngerman]{babel}
\usepackage[english]{babel}

\usepackage{amsmath}
\usepackage{graphicx}
\usepackage[colorlinks,urlcolor=black,citecolor=black,linkcolor=black]{hyperref}

\title{Conducting Psychoacoustic Experiments with the SoundScape Renderer}

\author{Matthias Geier, Sascha Spors}

\newcommand{\twoemailscombolink}[4][,]{\texttt{\{\href{mailto:#2#4}{#2}#1\href{mailto:#3#4}{#3}\}#4}}
\newcommand{\threeemailscombolink}[5][,]{\texttt{\{\href{mailto:#2#5}{#2}#1\href{mailto:#3#5}{#3}#1\href{mailto:#4#5}{#4}\}#5}}

\address{%
Address: Quality \& Usability Lab, Deutsche Telekom Laboratories, TU-Berlin,
Ernst-Reuter-Platz 7,
10587 Berlin\\
%E-Mail: \texttt{\{matthias.geier,sascha.spors\}@telekom.de}\\
%E-Mail: \twoemailscombolink{matthias.geier}{sascha.spors}{@telekom.de}\\
E-Mail: \threeemailscombolink{matthias.geier}{sascha.spors}{SoundScapeRenderer}{@telekom.de}\\
Web: \url{http://qu.tu-berlin.de}}

%\usepackage[nooneline,FIGTOPCAP,tight,sf,SF]{subfigure}
%\usepackage[tight,SF]{subfigure}
\usepackage[tight]{subfigure}
\setlength{\subfigbottomskip}{0pt}
\setlength{\subfigtopskip}{0pt}
%\setlength{\subfiglabelskip}{0pt}

\usepackage{emp}
\empaddtoTeX{%
\usepackage{times}
%\usepackage[T1]{fontenc}
\usepackage{amsmath}
%\usepackage{textcomp}
\newcommand{\fontnormal}{\normalfont\footnotesize\sffamily}
\newcommand{\fontsmall}{\normalfont\scriptsize\sffamily\itshape}
}

\ifpdf
% allow MPS files with extensions .1, .2, ... from metapost (via emp.sty)
\DeclareGraphicsRule{*}{mps}{*}{}
\fi

\usepackage{balance}

\begin{document}

% Block Diagrams: <<<
\begin{empfile}[emp/blockdiagrams]
\begin{empcmds}
  input boxes;
  u := 4mm;
  dotsize := 2pt;
  pair vspace; vspace := (0,u);
  pair hspace; hspace := (u,0);
  pair smallvspace; smallvspace := (0,.5u);
  pair smallhspace; smallhspace := (.5u,0);
  ahlength := 3.5bp; % length of arrow head (default 4 bp)

vardef arrowhead expr p =
  save q,e; path q; pair e;
  e = point length p of p;
  q = gobble(p shifted -e cutafter makepath(pencircle scaled 2ahlength))
  cuttings;
  (q rotated .5ahangle & reverse q rotated -.5ahangle -- (point .25 of q) --
  cycle)  shifted e
enddef;
def drawdotarrow expr p =
  draw point 0 of p withpen pencircle scaled dotsize;
  drawarrow p
enddef;

verbatimtex
% arguments to \genfrac: opening delimiter, closing del., width of line, style
% style: 0 \displaystyle, 1 \textstyle, 2 \scriptstyle, 3 \scriptscriptstyle
% long story short: to change the size, change the number!
\newcommand{\twolines}[2]{$\genfrac{}{}{0pt}{1}{\text{\itshape#1}}{\text{\itshape#2}}$}
etex

  verbatimtex \fontnormal etex;
\end{empcmds}
\begin{empdef}[wfs](0,0)
%boxit.in(btex  $\displaystyle \text{source}\atop\displaystyle\text{signal}$
%etex);
%boxit.in(btex  $\text{source}\atop\text{signal}$ etex);
boxit.in(btex \twolines{source}{signal} etex);
boxit.filter(btex Filter etex);

boxit.delay(btex \smash[b]{Delay} etex);
boxjoin(a.e + hspace = b.w);
boxit.weight(btex \smash[b]{Weight} etex);
circleit.plus(btex $+$ etex);
boxit.ls(btex \fontsmall loudspeaker etex);

boxjoin();

%boxit.other_ls(btex  $\text{other}\atop\text{loudspeakers}$ etex);
%boxit.other_ls(btex  $\genfrac{}{}{0pt}{2}{\text{other}}{\text{loudspeakers}}$
%etex);
boxit.otherls(btex \twolines{other}{loudspeakers} etex);

boxjoin(a.e = b.w);
boxit.distance(btex \twolines{source-loudspeaker}{distance} etex);
boxit.incidence(btex \twolines{angle of}{incidence} etex);
boxjoin();

%boxit.distance(btex \fontsmall distance etex);
%boxit.sourcepos(btex \twolines{source}{position} etex);
%boxit.lsposition(btex \twolines{loudspeaker}{position} etex);

boxit.othersources(btex \twolines{other}{sources} etex);

in.e + hspace = filter.w;
filter.e + 2hspace = delay.w;
delay.e + hspace = weight.w;

xpart otherls.c = xpart .5[filter.e,delay.w];
ypart otherls.n = ypart (filter.s - vspace);

%distance.e = incidence.w;

%xpart distance.c = xpart delay.c;
xpart .5[distance.c,incidence.c] = xpart .5[delay.c,weight.c];
%xpart distance.c = xpart delay.c;
ypart distance.s = ypart (delay.n + vspace);

%ypart sourcepos.s = ypart lsposition.s = ypart (distance.n + vspace);
%xpart (sourcepos.w - .5hspace) = xpart distance.c;
%xpart (lsposition.e + .5hspace) = xpart distance.c;

othersources.n + vspace = plus.s;

% ATTENTION: order of the drawing commands matters!
drawunboxed(in);
drawboxed(filter,delay,weight,plus);
drawunboxed(otherls,ls,distance,incidence,othersources);
%drawunboxed(sourcepos,lsposition);

drawarrow in.e -- filter.w;
drawarrow filter.e -- delay.w;
drawdotarrow .5[filter.e,delay.w] -- otherls.n;

drawarrow delay.e -- weight.w;
drawarrow weight.e -- plus.w;
drawarrow plus.e -- ls.w;

drawarrow distance.s -- delay.n;
drawarrow distance.s --weight.n;
%drawarrow sourcepos.s -- distance.n;
%drawarrow lsposition.s -- distance.n;

drawarrow incidence.s -- weight.n;

drawarrow othersources.n -- plus.s;

\end{empdef}

\begin{empdef}[binaural2](0,0)

boxit.in(btex \twolines{source}{signal} etex);
boxit.weight(btex \smash[b]{Weight} etex);
boxit.distance(btex \twolines{source-listener}{distance} etex);
boxit.filterone(btex Filter etex);
boxit.filtertwo(btex Filter etex);
boxit.incidence(btex \twolines{angle of}{incidence} etex);
boxit.headphones(btex \twolines{head-}{phones} etex);

circleit.plusone(btex $+$ etex);
circleit.plustwo(btex $+$ etex);

boxit.othersourcesone(btex \twolines{other}{sources} etex);
boxit.othersourcestwo(btex \twolines{other}{sources} etex);

in.e + hspace = weight.w;
weight.n + vspace = distance.s;

z1 = weight.e + hspace;

filterone.s - vspace = filtertwo.n;
x1 + xpart hspace = xpart filterone.w = xpart filtertwo.w;
y1 = ypart .5[filterone.s,filtertwo.n];

x1 = x2 = x3;
y2 = ypart filterone.c;
y3 = ypart filtertwo.c;

incidence.s = filterone.n + vspace;

filterone.e + hspace = plusone.w;
filtertwo.e + hspace = plustwo.w;

othersourcesone.s - vspace = plusone.n;
othersourcestwo.n + vspace = plustwo.s;

xpart headphones.w = xpart (plusone.e);
ypart headphones.w = y1;

drawunboxed(in);
drawboxed(weight,filterone,filtertwo,plusone,plustwo);
drawunboxed(distance,headphones,incidence,othersourcesone,othersourcestwo);

drawarrow in.e--weight.w;
drawarrow distance.s--weight.n;
drawarrow weight.e--z1--z2--filterone.w;
drawdotarrow z1--z3--filtertwo.w;
drawarrow filterone.e -- plusone.w;
drawarrow filtertwo.e -- plustwo.w;
drawarrow plusone.e -- plusone.e + hspace;
drawarrow plustwo.e -- plustwo.e + hspace;
drawarrow othersourcesone.s -- plusone.n;
drawarrow othersourcestwo.n -- plustwo.s;

pair oo;
oo = .5smallhspace;
drawarrow incidence.s-oo--filterone.n-oo;
path secondarrow;
secondarrow = incidence.s+oo--filtertwo.n+oo;
draw secondarrow cutafter reverse bpath.filterone;
drawarrow secondarrow cutbefore bpath.filterone;

\end{empdef}

\begin{empdef}[vbap2](0,0)

boxit.in(btex \twolines{source}{signal} etex);
boxit.weight(btex \smash[b]{Weight} etex);
boxit.distance(btex \twolines{source-loudspeaker}{distance} etex);
boxit.weightone(btex \smash[b]{Weight} etex);
boxit.weighttwo(btex \smash[b]{Weight} etex);
boxit.incidence(btex \twolines{angle of}{incidence} etex);
boxit.ls(btex \twolines{pair of}{loudspeakers} etex);

circleit.plusone(btex $+$ etex);
circleit.plustwo(btex $+$ etex);

boxit.othersourcesone(btex \twolines{other}{sources} etex);
boxit.othersourcestwo(btex \twolines{other}{sources} etex);

in.e + hspace = weight.w;
weight.n + vspace = distance.s;

z1 = weight.e + hspace;

weightone.s - vspace = weighttwo.n;
x1 + xpart hspace = xpart weightone.w = xpart weighttwo.w;
y1 = ypart .5[weightone.s,weighttwo.n];

x1 = x2 = x3;
y2 = ypart weightone.c;
y3 = ypart weighttwo.c;

incidence.s = weightone.n + vspace;

weightone.e + hspace = plusone.w;
weighttwo.e + hspace = plustwo.w;

othersourcesone.s - vspace = plusone.n;
othersourcestwo.n + vspace = plustwo.s;

xpart ls.w = xpart (plusone.e);
ypart ls.w = y1;

drawunboxed(in);
drawboxed(weight,weightone,weighttwo,plusone,plustwo);
drawunboxed(distance,ls,incidence,othersourcesone,othersourcestwo);

drawarrow in.e--weight.w;
drawarrow distance.s--weight.n;
drawarrow weight.e--z1--z2--weightone.w;
drawdotarrow z1--z3--weighttwo.w;

drawarrow weightone.e -- plusone.w;
drawarrow weighttwo.e -- plustwo.w;
drawarrow plusone.e -- plusone.e + hspace;
drawarrow plustwo.e -- plustwo.e + hspace;
drawarrow othersourcesone.s -- plusone.n;
drawarrow othersourcestwo.n -- plustwo.s;

pair oo;
oo = .5smallhspace;
drawarrow incidence.s-oo--weightone.n-oo;
path secondarrow;
secondarrow = incidence.s+oo--weighttwo.n+oo;
draw secondarrow cutafter reverse bpath.weightone;
drawarrow secondarrow cutbefore bpath.weightone;

\end{empdef}

\begin{empdef}[aap](0,0)
boxit.in(btex \twolines{source}{signal} etex);

boxjoin(a.e + hspace = b.w);
boxit.weight(btex \smash[b]{Weight} etex);
circleit.plus(btex $+$ etex);
boxit.ls(btex \fontsmall loudspeaker etex);

boxjoin();

boxit.otherls(btex \twolines{other}{loudspeakers} etex);

boxjoin(a.e = b.w);
boxit.distance(btex \twolines{source-loudspeaker}{distance} etex);
boxit.incidence(btex \twolines{angle of}{incidence} etex);
boxjoin();

boxit.othersources(btex \twolines{other}{sources} etex);

in.e + 2hspace = weight.w;

xpart otherls.c = xpart .5[in.e,weight.w];
ypart otherls.n = ypart (weight.s - vspace);

xpart .5[distance.c,incidence.c] = xpart weight.c;
ypart distance.s = ypart (weight.n + vspace);

othersources.n + vspace = plus.s;

% ATTENTION: order of the drawing commands matters!
drawunboxed(in);
drawboxed(weight,plus);
drawunboxed(otherls,ls,distance,incidence,othersources);

drawarrow in.e -- weight.w;
drawdotarrow .5[in.e,weight.w] -- otherls.n;

drawarrow weight.e -- plus.w;
drawarrow plus.e -- ls.w;

drawarrow distance.s --weight.n;

drawarrow incidence.s -- weight.n;

drawarrow othersources.n -- plus.s;

\end{empdef}

\end{empfile}
% On-the-fly metapost compilation:
\immediate\write18{cd emp && mpost --tex=latex blockdiagrams}
% >>>

\maketitle

\begin{abstract}
The \emph{SoundScape Renderer} (SSR) is a versatile software tool for
real-time spatial audio reproduction. It is capable of generating driving
signals for a variety of loudspeaker-based and headphone-based reproduction
methods.
Due to its extensive remote-control possibilities, it can be conveniently used
as an audio backend in psychoacoustic listening and conversational
experiments, for example in the area of spatial telephone conferencing.

The main features of the software are described and some aspects about its
architecture are presented. Different possibilities are shown how the SSR has
already been
used for conducting subjective tests.

The \emph{SoundScape Renderer}
%is \emph{Free and Open Source Software} and
can be downloaded from \url{http://tu-berlin.de/?id=ssr}
as \emph{Free and Open Source Software}.
\end{abstract}

\section{Motivation}

In comparison to traditional speech and audio listening tests, subjective
experiments in the context of spatial audio communication put higher demands on
the test setup. In addition to temporal and timbral attributes, also spatial
attributes have to be presented, controlled and analyzed.

For loudspeaker-based tests, the driving signal for each loudspeaker has to be
calculated in real-time if the sound sources' positions or other parameters
shall be changed interactively during the experiment.
One problem of using loudspeakers for listening tests is the exact placement of
the test subject withing the listening area and the controlled transition
between listening positions.
To achieve reproducible test conditions, often headphone-based reproduction is
chosen. The spatial impression is most natural when head-tracking is used to
compensate the test subject's head movements. This presentation
method -- which is commonly used for tests in the area of spatial telephone
conferencing -- is called \emph{dynamic binaural \mbox{(re-)}synthesis}.
Because of the necessity to compensate head movements, it is not possible to use
pre-computed audio files for playback and the output signals have to be
generated in real-time.
Additionally, in some adaptive test methods, the signals change dynamically
depending on the test subject's choices.

As a means of interaction between the test subject and the test setup often
specialized \emph{Graphical User Interfaces} (GUIs) are needed to realize
certain test methodologies.
For implementation and maintenance it is normally much easier to use separate
programs for the presentation of the audio stimuli and for displaying the test
GUI.
Both programs have to share a common interface to exchange control data, for
example via network sockets.

In the following sections a framework for spatial audio experiments is presented
which is able to cope with all aforementioned requirements.

\section{Technical Details}

The \emph{SoundScape Renderer} (SSR)~\cite{Geier2008SSR} is a tool for
\emph{object-based}
audio reproduction, i.e.\ input signals are
represented as sound objects with a given position in space and several other
parameters. The information for all sound objects is
stored in a so-called \emph{scene description}. Based on this scene description
the output channel signals of the desired reproduction system are generated in
real-time.

The SSR is written in C++ and compiled with the GNU compiler.
It is running on Linux and with some effort it can also be installed on
Mac~OS~X.

\subsection{Reproduction Methods}

The SSR is not limited to a single reproduction method or hardware setup.
Some of the available rendering modules are:

\begin{itemize}
\item loudspeaker-based:
\begin{itemize}
\item \emph{Wave Field Synthesis} (WFS)~\cite{Spors2008wfs}
\item \emph{Vector Base Amplitude Panning} (VBAP)~\cite{Pulkki97Virtual}
\item \emph{Ambisonic Amplitude Panning} (AAP)~\cite{Neukom2007}
\end{itemize}
\item headphone-based:
\begin{itemize}
\item \emph{HRTF-based Binaural Renderer}
\item \emph{Binaural Room Synthesis} (BRS)
\item \emph{Binaural Playback Renderer} (BPB)
\end{itemize}
\end{itemize}

\noindent When using the HRTF-based renderer, all sources share the same set of
\emph{Head Related Impulse Responses} (HRIRs) which are typically measured in a
dry, non-reverberant environment.
The BRS renderer uses \emph{Binaural Room Impulse Responses} (BRIRs) which
contain room information specific to the position where they were measured.
Therefore, each source needs its own set of BRIRs and the source positions
cannot be changed in the SSR.

The BPB renderer is not calculating the headphone signals in real-time but is
rather an audio file player with head-tracking capabilities.
Binaural signals for each head rotation in steps of one degree have to be
generated beforehand, the BPB renderer merely selects the appropriate channels
depending on the head orientation and does a crossfade if the orientation
changes.
This is useful for investigating algorithms for which no
real-time implementation exists, for example for rapidly moving sources.
The BPB renderer is still in development and not yet included in the public
release of the SSR.

\begin{figure}[tb]
\begin{center}
\subfigure[Binaural Rendering]{\empuse{binaural2}\label{fig:binaural}}\\[\baselineskip]
\subfigure[Wave Field Synthesis]{\empuse{wfs}\label{fig:wfs}}\\[\baselineskip]
\subfigure[Vector Base Amplitude Panning]{\empuse{vbap2}\label{fig:vbap}}\\[\baselineskip]
\subfigure[Ambisonic Amplitude Panning]{\empuse{aap}\label{fig:aap}}
\end{center}
\caption{Signal flow in different rendering modules}
\end{figure}

\subsection{Signal Processing}

All renderers are realized as a combination of a few basic processing units:
filtering (partitioned convolution), delay line and weighting.

The BRS and HRTF renderers -- figure~\ref{fig:binaural} -- are implemented
by convolving the input signal with impulse
responses corresponding to the left and right ear, dynamically chosen from a
database of impulse responses depending on the source angle with respect to the
listener position and
the head orientation of the listener. Before that, the source signal is
weighted with a value depending on its position to simulate distance
attenuation.

The WFS renderer -- figure~\ref{fig:wfs} -- is implemented by applying the
same pre-filtering
step to all input channels and then -- individually for each combination of
source and loudspeaker channel -- delaying and weighting the resulting signals
depending on the relative position and orientation of source and loudspeaker.

When using the VBAP renderer -- figure~\ref{fig:vbap} -- only two loudspeakers
are active for each sound source. A panning function is used to calculate the
weights of the two loudspeakers. Another weighting factor is applied to both
loudspeakers to account for the distance attenuation.

The AAP renderer -- figure~\ref{fig:aap} -- uses the principles of
\emph{Higher Order Ambisonics} (HOA) to calculate a panning function which
potentially employs all loudspeakers to reproduce one source.

All renderers use the same renderer base class with a straightforward interface
which makes it simple to implement new renderers.
It also provides parallel processing capabilities, which can be utilized by all
renderers.

\subsection{Audio Input/Output}

The \emph{JACK Audio Connection Kit} (JACK) is utilized for input and output of
audio data.
This allows for a great flexibility in routing between different audio
processing applications and the hardware ports.
Any audio application with JACK support as well as the physical inputs
of the sound card(s) can be used to provide the input
signals.
The SSR has been used with loudspeaker setups with up to 192 channels.

\subsection{Control}

The SSR can be controlled in different ways. On the one hand, it has a
built-in \emph{Graphical User Interface} (GUI) by means of which sound sources
can be moved around and their volume and other parameters can be changed
interactively.
In experimental setups, however, the built-in GUI is mostly in the background or
entirely deactivated.

On the other hand, there is a network interface to which a control application
can connect via a TCP/IP socket.
This socket interface has already been used by several GUIs written in
\emph{Python}, two GUIs written in \emph{ActionScript}, two GUIs written in
\emph{Matlab}, a remote-control created with \emph{Pure Data}, a GUI running on
a mobile phone with \emph{Android OS} and an \emph{iPhone} application.
Over its network interface, the SSR sends and receives text messages in a simple
XML format. This way the network messages can be extended in a very flexible way
and the network traffic can be analyzed easily in case of problems.

\begin{figure}[tb]
\begin{center}
%\includegraphics[scale=0.5]{screenshot14}
\includegraphics[width=\columnwidth]{screenshot14}
\end{center}
\caption{Screenshot of the test GUI from~\cite{geier2010:perceptual}, created
with \emph{Python}, \emph{GTK+} and \emph{glade}.}
\label{fig:RGTscreenshot}
\end{figure}

\section{Previous Experiments and Setups}

The modular architecture of the SSR allows its use in many different setups.
It has already been used in about a dozen psychoacoustic experiments.
Most of them were using headphone-based reproduction, but also experiments with
loudspeaker arrays have been conducted.

Binaural synthesis has been realized using \emph{Head Related Impulse Responses}
(HRIRs) measured in non-reverberant rooms as well as \emph{Binaural Room
Impulse Responses} (BRIRs) measured in more or less reverberant rooms.
The BRS renderer was also used with BRIRs calculated in \emph{Matlab} to render
virtual loudspeaker setups.

Audio playback was either done by the SSR or by an audio player created using \emph{Pure Data} (a.k.a.\ Pd). 
All components were connected by means of the JACK audio server.
In most experiments GUIs written in \emph{Python} were used to control the
SSR and Pd, but also GUIs written in \emph{Matlab} were used.

%The SSR was constantly rendering
%several sound sources by means of several pre-calculated BRIR sets.
%When the test subject changed the test condition in the GUI, Pd was changing the
%output channel which was connected to the SSR and thereby changing the currently
%active pair of BRIRs.

%The GUI has to communicate with the player
%application to start and stop playback at the desired times and with the SSR to
%select different conditions or change some reproduction parameters.

In the following paragraphs a few experiments using the SSR are briefly
presented and the test setup is explained.

A spatial telephone conference listening test has been conducted in
\cite{raake2010:listening} where recordings of three participants were played
back using Pd and the BRS renderer of the SSR was used to generate test
conditions as combination of different spatial presentation modes and different
telephone bandwidths. For each condition a BRIR set was calculated using
\emph{Matlab}. A GUI written in \emph{Python} was used to select the current
condition and start the playback.

An experiment about binaural monitoring of WFS systems has been done
in~\cite{Geier09:DAGA}. The same audio scene was presented using three different
methods: A real WFS system, a ``virtual'' WFS system simulated using the BRS
renderer with BRIR measurements of the real loudspeakers and a binaural rendering
using dry HRTFs. All three conditions were realized using the SSR and the
switching was done manually by the experimenter.

In \cite{geier2010:perceptual}, the perception of focused sources in WFS
has been evaluated. A ``virtual'' WFS system has been realized using the BRS
renderer. For each stimulus a BRIR set has been generated in \emph{Matlab}.
Figure~\ref{fig:RGTscreenshot} shows the GUI which was presented to the test
subjects.

A very similar setup was used in an experiment about the role of the precedence
effect for the perception of artifacts in WFS~\cite{wierstorf_precedence_2010}.
Again, \emph{Matlab} was used to generate BRIR sets to create a ``virtual'' WFS
system with the BRS renderer and head-tracked headphones.

The perceptual relevance of using fractional delays and of errors regarding
loudspeaker placement in WFS was investigated
in~\cite{ahrens2010:perceptual}. Except different BRIR sets, a different test
method and a different GUI, the test setup was very similar to the previously
mentioned experiment.

In addition to the aforementioned research papers, the SSR was also used in
several diploma theses.
In this context a perceptual comparison of WFS and HOA was done -- again using a
``virtual'' loudspeaker setup -- using the BRS renderer. Both the GUI and the
audio file playback was done in \emph{Matlab}.

Another diploma thesis was proposing different panning techniques for rapidly
moving sources in WFS. A listening test was done using the \emph{Binaural
Playback} (BPB) renderer. Audio playback was done by the SSR and a GUI written
in \emph{Python} was used to mute and un-mute stimuli according to the test
subject's selections.

\section{Conclusions and Future Work}

The \emph{SoundScape Renderer} (SSR) is a versatile tool for spatial audio
reproduction which is freely available as \emph{Open Source Software}.
It can be used in psychoacoustic experiments in the context of spatial audio
reproduction and transmission.
It can be connected to a variety of control applications using its network
socket interface.

Future plans include the implementation of
\emph{OpenDAFF}~\cite{wefers2010:opendaff} support for directional impulse
responses (\url{http://www.opendaff.org}) and the definition of a common scene
format for control and interchange of spatial audio scenes between different
reproduction systems.

Visit \url{http://tu-berlin.de/?id=ssr}
to get your own copy of the SSR
and feel free to contact
\href{mailto:SoundScapeRenderer@telekom.de}
{\texttt{SoundScapeRenderer@telekom.de}}
if you have questions or suggestions.
Any comments are greatly appreciated!

\balance

\bibliographystyle{unsrt}
\bibliography{itgspeech2010_SSR}

\end{document}

% vim:textwidth=80
